\documentclass[a4paper, report,notitlepage]{jlreq}
\usepackage{../takker}
\addbibresource{references.bib}
\title{Need for Nuclear power in Japan}
\author{Takuto Ishii}

\begin{document}
\maketitle

I think that Japan should use nuclear power. The use of nuclear power in Japan is essential for several compelling reasons.
Firstly, it is imperative to maintain and advance nuclear technology for future generations. Whether Japan decides to continue using nuclear power or not, it is crucial to pass on and develop nuclear technology for various reasons.
Decommissioning aging reactors and managing nuclear waste are ongoing tasks that require a skilled workforce.
Without a sustainable demand, such as restarting existing reactors or constructing smaller ones, there is a risk of losing expertise in this field. To ensure the capability to decommission reactors in the future, a certain number of operational reactors are necessary. Furthermore, nuclear power serves a vital role as a baseload power source. Japaneses renewable energy sources, while abundant in potential, face challenges in providing stable and consistent energy supply. Factors such as the scarcity of suitable flat land for solar power generation and limited regions with consistently strong winds hinder the widespread use of renewable energy sources. In order to maintain a stable energy supply and reduce greenhouse gas emissions, the use of cost-effective nuclear power is essential as a baseload power source. However, it is worth noting that recent developments in geothermal energy in Japan may provide an alternative for baseload power in the future. In conclusion, Japan should continue to use nuclear power for both the preservation and advancement of nuclear technology and as a reliable baseload power source. These two factors, in conjunction, make a strong case for the continued utilization of nuclear power in the Japanese energy landscape.

\nocite{*}
\printbibliography{}
\end{document}